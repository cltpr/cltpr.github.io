\documentclass[a4paper,11pt]{article}
\usepackage[french]{babel}
\usepackage[T1]{fontenc}
\usepackage[utf8]{inputenc}
\usepackage{lmodern}
\usepackage{microtype}
\usepackage{geometry}
\usepackage{qcm}
\usepackage{eurosym}
\usepackage{alterqcm}

\title{GITE DE L'ECURIEUX\\CONTRAT DE LOCATION SAISONNIERE\\d’un gîte familial écologique\\EVRES EN ARGONNE - MEUSE}
\begin{document}
\date{}
\maketitle

\center Entre les soussignés:\\
\flushleft 
Pierre CAILLET\\
5, rue de bellenette\\
55250 EVRES\\                                                                                         
Téléphone portable: 06 50 93 31 87 \\
Mail: clt.pr@protonmail.com\\
Site: www.ferme-de-l-ecurieux.fr\\
SIRET: 879 952 745 00013\\

\vspace{0.5cm}
Dénommés le bailleur d’une part,                            

\center Et:

\flushleft
Nom: \\
Prénom:\\
Téléphone:\\
Adresse:\\
Pièce d'identité (type et numéro):\\

\vspace{0.5cm}
Dénommé le locataire d'autre part.



\flushleft
Il a été convenu d'une location saisonnière au 5, rue de Bellenette 55250 EVRES, pour la période
\vspace{0.5cm}

du   .......   à .....    heures .....    (arrivée après 17h)
\vspace{0.5cm}

au   ........   à .....    heures .....   (départ avant 13h)
\vspace{0.5cm}

Nb de personnes (adultes et enfants\footnote{dormant dans un lit normal}): 
\vspace{0.5cm}

\newpage{}




\vspace{0.5cm}


\begin{tabbing}
  
\hspace{6cm}\=  										\hspace{1cm}\=   	\hspace{4cm}\= 	\hspace{2cm}\= 		\hspace{2cm}\=	\kill


Deux premières nuits\>220\euro		\>		\>=			\>220\>\euro\\
Préparation du gîte\>80\euro		\>		\>=			\>80\>\euro\\
Nuits supplémentaires \>55\euro\> x ....nuits		\>=			\>......\>\euro\\
Linge de lit (lits non faits) :	\>5\euro\>x ....lit(s) \>=  	\>......\>\euro	\\
Linge de lit (lits faits) :	\>10\euro\>x ....lit(s) \>=  	\>......\>\euro	\\
3 chambres supplémentaires \>50\euro\> x ....nuits		\>=			\>......\>\euro\\
Utilisation du bain nordique \>50\euro		\>		\>=			\>......\>\euro\\
Taxe de séjour \>0,80\euro		\>x....pers x ....nuits		\>=	\>......\>\euro\\

Total loyer final    										\>						\>  					\>=  							\>......					\>\euro			\\ 

 \end{tabbing}

Nota: la taxe de séjour n'est pas due pour les personnes mineures. Pour les locations par plates-formes numériques, ces dernières prélèvent automatiquement.


\vspace{0.5cm}

  
Si vous souhaitez régler par chèque, un chèque d'arrhes de 50\% du loyer final (à l'ordre du bailleur: Pierre CAILLET)  doit accompagner le présent contrat correctement rempli et signé. 
Un transfert par virement bancaire est également possible\footnote{IBAN: FR76 1027 8020 0100 0211 0250 155}. Dans ce cas le loyer final et la caution devront être arrivés sur le compte du bailleur avant la date d'entrée dans les lieux.
Dans le cas d'un réglement en espèces à l'entrée dans les lieux, la réception du contrat signé reste absolument nécessaire pour valider la réservation. La caution devra également être présentée en espèces dans ce cas.


\vspace{0.5cm}

A l'entrée dans les lieux, il sera demandé au locataire:
\begin{itemize}
\item le solde du loyer, par chèque ou en espèces (sauf location par plates-formes numériques).
\item le réglement des éventuels suppléments
\item une caution de 450 euros, par chèque ou en espèces (établissement d'un reçu)
\end{itemize}
En cas de réglement et caution par chèque, ils devront tous être émis par la même personne que le chèque d'arrhes.

\vspace{0.5cm}

Le locataire déclare accepter les conditions générales de location annexées au présent contrat, ainsi qu'avoir pris connaissance de l'état descriptif des lieux.

\vspace{0.5cm}

Fait à .................................... le ...............................................

\vspace{0.5cm}

Le Bailleur : \hspace{3cm}Le Locataire : 

\hspace{5.2cm}mention manuscrite "lu et approuvé" à écrire


\newpage{}




\begin{center}
\section*  { ANNEXE: CONDITIONS GENERALES  DE LOCATION}
\end{center}


\tiny



La présente location est faite aux conditions ordinaires et de droit en pareille matière et notamment à celles ci-après, que le locataire s’oblige à exécuter, sous peine de tous dommages et intérêts et même de résiliations des présentes, si bon semble au mandataire et sans pouvoir réclamer la diminution du loyer.
Le contrat est réputé validé à la réception d’un exemplaire valide et signé du présent contrat et des règlements correspondants.

\begin{enumerate}

\item Les heures d’arrivée et de départ sont inscrites  au contrat par vos soins permettant notre présence pour l’accueil, la remise et la reprise des clefs. L'entrée dans les lieux ne peut se faire avant 17h, la sortie ne peut se faire après 13h.
Des modifications sont possibles par accord mutuel entre le bailleur et le locataire,
 


\item Il est interdit de fumer à l’intérieur du gîte. Les mégots des cigarettes fumées à l'extérieur doivent être jetés dans les cendriers.

\item Il est convenu qu'en cas de désistement \\


\begin{itemize}
\item du locataire, quelque soit le motif:
\subitem à plus d'un mois avant la prise d'effet du bail, les arrhes seront reversés au locataire
\subitem à moins d'un mois avant la prise d'effet du bail, la totalité des arrhes seront acquises au bailleur rendant de fait le logement libre
\item du bailleur, quelque soit le motif:
\subitem à plus d'un mois avant la prise d'effet du bail, les arrhes seront reversés au locataire
\subitem à moins d'une semaine avant la prise d'effet du bail, dans les sept jours suivant le désistement, il est tenu de verser le double des arrhes au locataire.
\end{itemize}



\item Si un retard de plus de quatre jours par rapport à la date d’arrivée prévue n’a pas été signalé par le locataire, le bailleur pourra de bon droit, essayer de relouer le logement tout en conservant la faculté de se retourner contre le preneur.



\item Obligation est faite au locataire d’occuper les lieux personnellement, de les habiter “en bon père de famille responsable” et de les entretenir.

\item L'état descriptif des lieux est à rendre dans les deux heures suivant l'arrivée du locataire. Passé ce délai il est réputé accepté et sans remarques.


\item Obligation est faite au locataire de veiller à ce que la tranquillité du voisinage ne soit pas troublée par son fait, celui des personnes qui l'accompagnent ou de son animal domestique.

\item La capacité d'accueil du gîte ne doit pas être dépassée lors des nuits.



\item Les locaux sont loués meublés avec matériel de cuisine, vaisselle, verrerie, draps, couvertures et oreillers (draps en option payante), tels qu’ils sont dans l’état descriptif à vérifier à l'arrivée.
S’il y a lieu, le bailleur ou son représentant seront en droit de réclamer au locataire à son départ et dans les 3 semaines suivantes:\\
- la valeur totale au prix de remplacement des objets, mobiliers ou matériels cassés, fêlés, ébréchés ou détériorés et de ceux dont l’usure dépasserait la normale pour la durée de la location\\
- le prix du nettoyage des couvertures ou couettes rendues sales\\
- une indemnité pour les détériorations de toute nature concernant tous les éléments constitutifs du logement, les rideaux, papiers peints, plafonds, sols, vitres, literie, etc...\\



\item Le locataire doit être assuré contre les risques locatifs (incendie, dégât des eaux). 
Le bailleur s'engage à assurer le logement contre les risques locatifs pour le compte du locataire, ce dernier ayant l'obligation de lui signaler  tout sinistre survenu dans le logement, ses dépendances ou accessoires avant le départ.

\item Si cette option a été choisie, l'utilisation du bain nordique familial est de l'entière responsabilité du locataire. Les enfants dans le jardin doivent être sous surveillance en permanence. 



\item L'utilisation du lit superposé supérieur ne convient pas aux enfants de moins de 6 ans.

\item Le locataire déclare accepter le principe d'utilisation de toilettes écologiques à litière bio-maîtrisée.

\item En fonction de l’état des lieux et des déclarations en fin de contrat, le locataire s’engage à régler tous les frais, casse, perte, dégradations etc... résultant du fait de son séjour, y compris ceux découverts après son départ.
Ceux-ci feront l'objet d'une note de frais pouvant être, en toute bonne foi, contestée. 
Le dépôt de garantie est retourné par courrier sous 1 mois maximum, éventuellement diminué des frais évoqués supra.
Dans le cadre des économies d’énergie, le locataire s’engage également à user normalement des énergies (charges gratuites).
Un supplément de frais pourra être facturé en apportant la preuve de l’utilisation anormale.

\item Election de domicile pour tout problème : au domicile du bailleur. Toute action en justice se fera auprès du tribunal de Bar le Duc.

\item Le locataire ne pourra s’opposer à la visite des locaux, lorsque le propriétaire ou son représentant en feront la demande circonstanciée.

\item Les animaux domestiques sont accueillis sous la responsabilité des locataires. Il est interdit au locataire de laisser un animal seul dans le logement. Pour des animaux autres que chiens et chats, l'accord préalable du bailleur est nécessaire.

\end{enumerate}

\hspace{10cm}
Version: \date{\today}


\normalsize

\newpage{}

ETAT DESCRIPTIF DES LIEUX
 
Cet état descriptif est aussi un état des lieux sommaire. Il doit être signé à la remise des clés. En cas de réserve à indiquer, le locataire les indique sur le document avant signature.

\vspace{0.25cm}


Clés
\begin{itemize}
\item deux clés de la porte d'entrée
\end{itemize}


\vspace{0.25cm}

Jardin et terrasse
\begin{itemize}
\item une table de jardin et 8 chaises, un banc bois (état moyen), trois fauteuils de jardin
\item une maison d'enfant
\item un bain nordique bois avec tuyau inox et escalier bois
\item un barbecue avec grille
\item un composteur plastique
\end{itemize}

\vspace{0.25cm}

Cuisine
\begin{itemize}
\item meuble bar avec deux chaises hautes 
\item cafetière à piston et bouilloire
\item frigo, four, four micro ondes
\item grille pain
\item ensemble raclette
\item piano droit
\end{itemize}

\vspace{0.25cm}

Séjour
\begin{itemize}
\item deux canapés
\item fauteuil noir
\item TV murale avec box orange
\item piano droit
\item deux tables et dix chaises, un banc, une lampe de de chevet
\end{itemize}

\vspace{0.25cm}

Salle de bains rez de chaussée
\begin{itemize}
\item toilette à litière bio maîtrisée
\item douche avec cabine verre, ensemble thermostatique
\item miroir, patère
\item meuble 1 vasque avec deux tiroirs
\item poubelle plastique
\end{itemize}

\vspace{0.25cm}
Couloirs
\begin{itemize}
\item escalier bois. Dessous: seau, serpillière, balai, pelle. Rideau sur tringle.
\item matériel bébé: parc, chaise haute, table à langer
\item à l'étage: meuble penderie avec cintres
\end{itemize}

Dans les chambres, chaque lit dispose d'un matelas avec housse de protection, d'un oreiller avec housse de protection (deux pour les lits doubles),  d'une couverture et d'un couvre lit.

Si cette option a été choisie, le linge de lit comprend, par lit équipé, un drap housse, un drap, une taie d'oreiller (deux pour les lits doubles).
\vspace{0.5cm}


\vspace{0.25cm}

Première chambre
\begin{itemize}
\item lit deux places, matelas et sommier 140*190 cm
\item un lit superposé avec sommiers, deux matelas 90*190 cm
\item deux tables de nuits, deux lampes de chevet
\item commode trois tiroirs
\item fenêtre PVC, tablette et volets bois battants
\end{itemize}

\vspace{0.5cm}

Petite chambre
\begin{itemize}
\item lit deux places, matelas et sommier 140*190 cm
\item deux tables de nuits, deux lampes de chevet
\item lit bébé avec matelas et tapis d'éveil
\item commode trois tiroirs
\item fenêtre PVC, tablette et volets bois battants
\end{itemize}


\vspace{0.25cm}

Grande chambre
\begin{itemize}
\item un lit superposé avec sommiers, deux matelas 90*190 cm
\item deux lits une place, matelas et sommier 90*190 cm
\item trois tables de nuit, trois lampes de chevet
\item meuble bas deux portes et deux tiroirs
\item fenêtre PVC, tablette et volets bois battants
\end{itemize}
\vspace{0.25cm}

Salle de bains étage
\begin{itemize}
\item un meuble double vasque avec deux tiroirs, deux miroirs avec éclairage, deux robinets mitigeurs
\item une baignoire avec robinet mitigeur, douche et colonne de douche, tapis de bain
\item deux tapis de sol, une poubelle
\end{itemize}

\vspace{0.5cm}

Toilettes étage
\begin{itemize}
\item toilette à litière bio maîtrisée
\item lave mains avec robinet mitigeur
\item poubelle
\end{itemize}

\vspace{0.25cm}

L'ensembles des sols, murs et plafond présente un état proche du neuf, toute détérioration constatée lors de l'entrée dans les lieux doit être consignée sur ce document.

\vspace{0.25cm}

Nombre de lignes annotées à la main: ..........................

\vspace{0.25cm}

Fait à .................................... le ...............................................

\vspace{0.25cm}

Le Bailleur : \hspace{3cm}Le Locataire : 

\hspace{5.2cm}mention manuscrite "lu et approuvé" à écrire


\newpage{}


\end{document}
